\documentclass{article}
\usepackage{amsmath}
\usepackage{graphicx}
\usepackage{caption}
\usepackage{float}
\usepackage{cleveref} % Added cleveref package
\usepackage{longtable} % Added longtable package
\usepackage{bicaption}
\usepackage{subcaption}

% Custom names for cref
\crefname{equation}{equation}{equations}
\crefname{figure}{figure}{figures}
\crefname{table}{table}{tables}
\crefname{section}{section}{sections}

\title{pandoc-tex-numbering Test Document}
\begin{document}
\maketitle
\section{Introduction}
This document is created to test the pandoc numbering functions for equations, figures, and tables. \Cref{sec:equations} (cref) and \cref{sec:figures_tables} (cref) provide examples of equations, figures, and tables.

\section{Equations}

\subsection{Multiline Equations\label{sec:equations}}
Here is a multiline equation using the \texttt{aligned} environment:
\begin{equation}
    \begin{aligned}
        f(x) &= x^2 + 2x + 1 \\
        g(x) &= \sin(x)
    \end{aligned}
    \label{eq:quadratic}
\end{equation}

Here is another multiline equation using the \texttt{align} environment, with the third equation not numbered:

\begin{align}
    a &= b + c \label{eq:align1} \\
    d &= e - f \label{eq:align2} \\
    g &= h \nonumber \\
    i &= j + k \label{eq:align3}
\end{align}

\subsection{Single Line Equations}
Here is a single line equation:
\begin{equation}
    E = mc^2 \label{eq:einstein}
\end{equation}

\subsection{References to Equations}
\Cref{eq:quadratic} (cref) represents a quadratic function. Equation \ref{eq:align1} (ref) and \cref{eq:align2} (cref) are simple algebraic equations. \Cref{eq:einstein} (cref) is Einstein's mass-energy equivalence. 
% TODO: You can also refer to equations at once like: \cref{eq:quadratic,eq:align1,eq:align2,eq:einstein}.

\section{Figures and Tables\label{sec:figures_tables}}

\subsection{Figures}
\Cref{fig:example} (cref) shows an example image with two subfigures: figure \ref{fig:subfig1} (ref) and \cref{fig:subfig2} (cref).

\begin{figure}[H]
    \begin{subfigure}[b]{0.5\textwidth}
        \centering
        \includegraphics[width=0.5\textwidth]{./example-image}
        \caption{Subfigure 1}
        \label{fig:subfig1}
    \end{subfigure}
    \begin{subfigure}[b]{0.5\textwidth}
        \centering
        \includegraphics[width=0.5\textwidth]{./example-image}
        \caption{Subfigure 2}
        \label{fig:subfig2}
    \end{subfigure}
    \caption[short caption]{Example Figure}
    \label{fig:example}
\end{figure}

\subsection{Tables}
\Cref{tab:example} (cref) shows an example table.

\begin{table}[H]
    \centering
    \begin{tabular}{|c|c|c|}
        \hline
        Column 1 & Column 2 & Column 3 \\
        \hline
        Data 1 & Data 2 & Data 3 \\
        Data 4 & Data 5 & Data 6 \\
        \hline
    \end{tabular}
    \caption{Example Table}
    \label{tab:example}
\end{table}

\begin{table}[H]
    \centering
    \begin{tabular}{|c|c|c|}
        \hline
        Column 1 & Column 2 & Column 3 \\
        \hline
        Data 1 & Data 2 & Data 3 \\
        Data 4 & Data 5 & Data 6 \\
        \hline
    \end{tabular}
    % \caption{Example Table without Label}
    % \label{tab:example}
\end{table}

\subsection{References to Figures and Tables}
\Cref{fig:example} (cref) is an example image. Table \ref{tab:example} (ref) is an example table. 

\section{Conclusion}
This document includes examples of multiline equations, single line equations, figures, and tables with labels and references to test the pandoc numbering functions. For more details, see \cref{sec:equations} (cref) and \cref{sec:figures_tables} (cref).

\end{document}