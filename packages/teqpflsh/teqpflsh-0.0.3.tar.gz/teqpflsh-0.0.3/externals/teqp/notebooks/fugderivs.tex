\documentclass{article}
\usepackage{cancel}
\newcommand{\deriv}[3]{\ensuremath{\left(\frac{\partial #1}{\partial #2}\right)_{#3}}}
\usepackage[hmargin=1in]{geometry}
\begin{document}
In the isochoric formalism, the fugacity coefficient array can be obtained by the gradient of the residual Helmholtz energy density (which is a scalar) and the compressibility factor $Z$  (which is also a scalar) in terms of the temperature $T$ and the molar concentration vector $\vec\rho$:
\begin{equation}
	\ln\vec\phi = \frac{1}{RT}\frac{\partial \Psi^r}{d\vec\rho} - \ln(Z)
\end{equation}

\textbf{Easy}: temperature derivative at constant molar concentrations (implying constant volume and molar composition)
\begin{equation}
	\deriv{ \ln\vec\phi}{T}{\vec\rho} = \frac{1}{RT}\frac{\partial^2 \Psi^r}{\partial \vec\rho\partial T} + \frac{-1}{RT^2}\deriv{\Psi^r}{\vec\rho}{T} - \frac{1}{Z}\deriv{Z}{T}{\vec\rho}
\end{equation}

\textbf{Medium}: molar density derivative at constant temperature and mole fractions
\begin{equation}
	\deriv{ \ln\vec\phi}{\rho}{T,\vec x} = \frac{1}{RT}\frac{\partial^2 \Psi^r}{\partial \vec\rho\partial \rho}  - \frac{1}{Z}\deriv{Z}{\rho}{T,\vec x}
\end{equation}
\begin{equation}
	Z = 1+\rho\deriv{\alpha^r}{\rho}{T}
\end{equation}
\begin{equation}
\deriv{Z}{\rho}{T,\vec x} = \rho\deriv{^2\alpha^r}{\rho^2}{T} + \deriv{\alpha^r}{\rho}{T}
\end{equation}

Back to basics, for a quantity $\chi$ that is a function of $T$ and $\vec\rho$, and then the derivative taken w.r.t. density at constant temperature and mole fractions:
\begin{equation}
	\deriv{\chi}{\rho}{T, \vec x} = 	\deriv{\chi}{T}{\vec \rho}\cancelto{0}{\deriv{T}{\rho}{T}} + \sum_i\deriv{\chi}{\rho_i}{T, \rho_{j\neq i}}\deriv{\rho_i}{\rho}{T,\vec x}
\end{equation}
with $\rho_i =x_i\rho$
\begin{equation}
\deriv{\rho_i}{\rho}{T, \vec x} = x_i
\end{equation}
thus
\begin{equation}\
	\deriv{\chi}{\rho}{T, \vec x} =  \sum_i\deriv{\chi}{\rho_i}{T, \rho_{j\neq i}}x_i
\end{equation}

and following the pattern yields
\begin{equation}
\frac{\partial^2 \Psi^r}{\partial \vec\rho\partial \rho} = 	\sum_i\deriv{\frac{\partial \Psi^r}{d\vec\rho} }{\rho_i}{T, \rho_{j\neq i}}x_i
\end{equation}
where the big thing is the Hessian of the residual Hessian matrix of the residual Helmholtz energy density.  This uses terms that are already developed.

\textbf{Medium+}: Volume derivative, based on the density derivative

\begin{equation}
\deriv{ \ln\vec\phi}{v}{T,\vec x} = \deriv{ \ln\vec\phi}{\rho}{T,\vec x}\deriv{ \rho}{v}{}
\end{equation}
\begin{equation}
\deriv{\rho}{v}{} = -1/v^2 = -\rho^2
\end{equation}

\textbf{Hard}: mole fraction derivatives (this results in a matrix rather than a vector)
\begin{equation}
	\deriv{ \ln\vec\phi}{\vec x}{T,\rho} = ?
\end{equation}

The first term is conceptually tricky. Again, considering a generic quantity $\chi$
\begin{equation}
	\deriv{\chi}{x_i}{T, \rho,  x_{j\neq i}} = \deriv{\chi}{T}{\vec \rho}\cancelto{0}{\deriv{T}{x_i}{T,\rho,x_{j\neq i}}} + \sum_i\deriv{\chi}{\rho_i}{T, \rho_{j\neq i}}\deriv{\rho_i}{x_i}{T,\rho, x_{j\neq i}}
\end{equation}
yields
\begin{equation}
	\deriv{\chi}{x_i}{T, \rho,  x_{j\neq i}} = \rho \sum_i\deriv{\chi}{\rho_i}{T, \rho_{j\neq i}}
\end{equation}
so the first part becomes
\begin{equation}
	\deriv{\frac{\partial \Psi^r}{d\vec\rho}}{x_i}{T, \rho,  x_{j\neq i}} = \rho \sum_i\deriv{\frac{\partial \Psi^r}{d\vec\rho}}{\rho_i}{T, \rho_{j\neq i}}
\end{equation}
or
\begin{equation}
	\deriv{^2\partial \Psi^r}{\vec\rho \partial \vec x}{T, \rho} = \rho H(\Psi^r)
\end{equation} 
which is somewhat surprising because the order of derivatives with respect to composition and density doesn't matter, as the Hessian is symmetric

The second part, from derivatives of $\ln Z$, with $Z$ given by
\begin{equation}
	Z = 1+\rho\deriv{\alpha^r}{\rho}{T, \vec x}
\end{equation}
yields
\begin{equation}
\deriv{\ln Z}{x_i}{T,\rho,x_{k \neq j}} = \frac{1}{Z}\deriv{Z}{x_i}{T,\rho,x_{k \neq i}}
\end{equation}
which results in a vector because you have 
\begin{equation}
\deriv{Z}{x_i}{T,\rho,x_{k \neq i}} = \rho \deriv{^2\alpha^r}{\rho\partial x_i}{T}
\end{equation}


\end{document}