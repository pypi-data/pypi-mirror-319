\documentclass{article}
\usepackage[utf8]{inputenc}
\usepackage{hyperref}
\usepackage{listings}
\usepackage{xcolor} % For custom colors
\usepackage{sectsty} % For customizing section fonts
% \subsectionfont{\raggedright} % Allow line breaks and ragged right alignment             
\usepackage{geometry}
% \geometry{margin=1in}
                
\title{Functions of the TexTOM module}
\author{Moritz Frewein}
\begin{document}
\maketitle
\label{toc}
\tableofcontents

% Define a custom style for docstrings
\lstdefinelanguage{docstring}{
    basicstyle=\ttfamily\small, % Monospaced font
    backgroundcolor=\color[HTML]{F5F5F5}, % Light gray background
    frame=single, % Border around the docstring
    rulecolor=\color[HTML]{D6D6D6}, % Border color
    keywordstyle=\color{blue}, % Optional: color for keywords
    breaklines=true, % Wrap long lines
}
                
\subsectionfont{\large\ttfamily\raggedright}

\section{Introduction}

\subsection{Texture Tomography}

Motivation and Output

Description of the experiment

Get in touch with Beamlines

\subsection{Installation}

TexTOM was written and tested in Python 3.11 and in principal requires only a python installation (3.9 to 3.12) and a terminal.
It can be used via scripts, jupyter notebooks or in iPython mode through the terminal.

The TexTOM core for reconstructions currently depends on the packages Scipy, Numba, H5py and Orix.
The recommended installation using the full pipeline, including also data integration and sample alignment requires the installation of pyFAI and Mumott.

We recommend creating a virtual venv or conda environment and installing the package via pip:

\begin{verbatim}
    python -m venv .venv/textom
    source .venvs/textom/bin/activate
\end{verbatim}
or
\begin{verbatim}
    conda create --name textom python=3.11
    conda activate textom
\end{verbatim}
then
\begin{verbatim}
    pip install textom
\end{verbatim}

It can also be built from source: 
\url{https://gitlab.fresnel.fr/textom/textom/}.

To start TexTOM in iPython mode, make sure your environment is activate and type textom.
All TexTOM core functions (\ref{sec:functions}) will be available in the namespace.

You can also import them into a script or jupyter notebook:
\begin{verbatim}
    from textom import *
\end{verbatim}

\section{Configuration}
After installing or updating TexTOM, we recommend opening the configuration file primarily to set how many CPUs your machine has
for data processing. Type textom\_config in your terminal and it will open the config file in your standard
text editor. A standard config file will look like the following:
\begin{verbatim}
    import numpy as np # don't delete this line
    ##################################################
    
    # Define how many cores you want to use 
    n_threads = 8 
    
    # Choose if you want to use a GPU for integration and alignment
    use_gpu = True
    
    # Choose your precision
    # recommended np.float64 for double or np.float32 for single precision
    data_type = np.float32
\end{verbatim}
After making your changes, you can save the file and close it.
If you plan to use a GPU for integration, you need to additionally install OpenCL, check the pyFAI documentation for
further information (sec. \ref{sec:integration}).

\section{Workflow}

\subsection{Data acquisition}
Recording data for texture tomography is a great challenge and can only be done at appropriate synchrotron beamlines.
This package contains a few scripts for the experiments but we recommend contacting a beamline scientist experienced in tensor/texture tomography or 
3D-XRD in order to create acquisition scripts suitable for the beamline.

Beamlines with experience in tensor/texture tomography:

ID13, ID15A at ESRF, Grenoble

\subsection{Data integration}\label{sec:integration}
The first step in data processing is integration, i.e. azimutal rebinning ("caking") of the carthesian 2D data.
Here we rely on the pyFAI package (\url{https://pyfai.readthedocs.io}).
This part already requires good knowledge of your data, as you do not want to miss any peaks when choosing the
integration range. We recommend to do a test-integration during the experiment, to set up the correct
.poni-file which is needed for the integration. This file defines the geometry of the experiment and can be
created using the command pyFAI-calib. Make sure to also collect the correct detector mask and optionally files
for flatfield and darkcurrent correction.

To start the integration, in your terminal navigate to a directory which will further contain all 
textom analysis data (further labelled sample\_dir).
\begin{verbatim}
    cd /path/to/textom/sample_dir
\end{verbatim}
Then start textom by typing textom in your terminal.
You can start the integration using the command integrate(), upon which a file containing all necessary
parameters will open:
\begin{verbatim}
    ############ Input ###############
    path_in = 'path/to/your/experiment/overview_file.h5'
    h5_proj_pattern = 'mysample*.1'
    h5_data_path = 'measurement/eiger'
    h5_tilt_angle_path = 'instrument/positioners/tilt' # tilt angle
    h5_rot_angle_path = 'instrument/positioners/rot' # rotation angle
    h5_ty_path = 'measurement/dty' # horizontal position
    h5_tz_path = 'measurement/dtz' # vertical position
    h5_nfast_path = None # fast axis number of points, None if controt
    h5_nslow_path = None # slow axis number of points, None if controt
    h5_ion_path = 'measurement/ion' # photon counter if present else None
    
    # Integration mode
    mode = 2 # 1: 1D, 2: 2D, 3: both
    
    # parallelisation
    n_tasks = 8
    cores_per_task = 16
    
    # Parameters for pyFAI azimuthal integration
    rad_range = [0.01, 37] # radial range
    rad_unit = 'q_nm^-1' # radial parameter and unit ('q_nm^-1', ''2th_deg', etc)
    azi_range = [-180, 180] # azimuthal range in degree
    npt_rad = 100 # number of points radial direction
    npt_azi = 120 # number of points azimuthal direction
    npt_rad_1D = 2000 # number of points radial direction
    int_method=('bbox','csr','cython') # pyFAI integration methods
    poni_path = 'path/to/your/poni_file.poni'
    mask_path = 'path/to/your/mask.edf'
    polarisation_factor= 0.95 # polarisation factor, usually 0.95 or 0.99
    flatfield_correction = None
    solidangle_correction = True
    darkcurrent_correction = None
    ##############################
\end{verbatim}
The first part contains information about your data. We assume that these are stored in .h5 files as common practice
at the ESRF. The first line is the overview file that contains links to all datasets. In the second line you can
specify which files should be integrated using a pattern with a * serving as a placeholder for other characters.
In the following there are the .h5 internal paths to the necessary metadata for TexTOM, which will be carried into the
integrated files. h5\_nfast\_path and h5\_nslow\_path are only relevant if the experiment was performed in scanning mode,
upon which all data of one projection will be in the same data array with the horizontal and vertical position not specified.
If the experiment was performed in continuous rotation mode, these parameters can be set to None.
The last parameter is optional for the measurement of an ionisation chamber or diode, which records the incoming photon
flux during the respective measurement.

Then choose the integration mode, 2D is required for TexTOM, 1D can be done additionally e.g. for diffraction tomography.

In the next block declare on how many CPUs you want to work parallely, the n\_tasks specifies how many files will be integrated
at the same time, cores\_per\_task means how many CPUs work on each task.

The last block are parameters for pyFAI, of particular importance are the radial range, which should cover your peaks
and the number of points (npt\_rad), which should be enough to resolve the individual peaks (although the code will
also handle overlapping peaks or peaks which are in a single bin to the cost of some information loss).
The required angular resolution depends on the sharpness of the features in the data in azimutal direction,
keep in mind that it is recommended to use a similar angular resolution for the construction of orientation
distribution functions and diffractlets, where the computation time will scale with the power of 3 of the number
of angular sampling points npt\_azi.  
Furthermore, point to the files you received from your beamline and specify angular resolution etc.

\subsection{Alignment}

Data is aligned fully automatically using the function:
\begin{verbatim}
    align_data( 
        pattern='.h5', sub_data='data_integrated', 
        q_index_range=(0,5), q_range = False,
        mode='optical_flow', crop_image=False, 
        regroup_max=16,
        redo_import=False, flip_fov=False, 
        align_horizontal=True, align_vertical=True,
        pre_rec_it = 5, pre_max_it = 5,
        last_rec_it = 40, last_max_it = 5,
          )
\end{verbatim}

The first step of the alignment is the sorting of the data. 
Go to the data\_integrated or data\_integrated\_1d directory created by the integration script
and make sure that all .h5 files are valid datasets, which you 
want to use for the reconstruction (other file extensions will be ignored). 
Move files that you don't want to use to a subfolder (e.g. named excluded).
The program uses all data in the sub\_data directory with pattern in the filename.
By default it uses data in data\_integrated/, you can use others by typing e.g. align\_data(sub\_data='data\_integrated\_1d')

Next, choose the q-range you want to use for alignment. You can use indices in the to restrain the q-values
using the q\_index\_range parameter or give a q-range directly in the units specified in the radial\_units
field in the data (this parameter has priority if specified). TexTOM will average over all data in this range 
and treat them as scalar tomographic data for alignment. We recommend using either the SAXS region of 
the sample or a bright peak with little azimutal variation.

TexTOM uses the alignment code from the Mumott tensor tomography package, which contains 2 pipelines.
By default we use the optical flow alignment, but you can choose phase matching alignment in the parameters.
If you want to crop the projections, set the crop\_image parameter to the desired borders (e.g. ((0,-1),(10,-10))
for the full image in x-direction, while cropping 10 points at the top and bottom)
Take note that cropping only works with the phase matching alignment, which will be chosen automatically if 
crop\_image is defined.

The textom alignment pipeline will downsample the data by combining blocks of 2x2 pixels until arriving at the 
sampling defined by regroup\_max, by default 16, corresponding to a downsampling to blocks of 16x16 pixels.
Then the alignment will start at the lowest sampling, take the found values and proceed to the next highest until it reaches the
original sampling. This approach has proven efficient, but can be omitted by setting regroup\_max=1.

For the remaining parameters see the description further down.

When you start the alignment, it will open a file labelled geometry.py, which contains information about the
experimental setup. Most parameters are equivalent to the Mumott notation (\url{https://mumott.org/tutorials/inspect_data.html#Geometry}),
which defines the arrangement of sample, detector, rotation and tilt angles.
In addition, you need to define beam diameter, step size and scanning mode.

When you close and save the file, it will be automatically stored in sample\_dir/analysis/geometry.py and in the following,
this file will be used. You can also create a geometry file in sample\_dir/analysis/ prior to starting the alignment,
then this file will directly be used (e.g. when you have several samples from the same beamtime, copy the geometry file after
defining it for the first sample.). The default values are given for the configuration published in Frewein et al. IUCRJ (2024).

After aligning, function will create the file analysis/alignment\_result.h5 in the sample directory, which contains the shifts found
in the process. Refer to this file for checking sinograms and tomograms after alignment.
You can also use the funciton check\_alignment\_consistency() to check if there are projections wich deviate from the
model. Inspect them and their agreement with the data using check\_alignment\_projection(g), where g is an integer number
corresponding to the projection number. This number is assigned after sorting the data files alphabetically.
The x-axis label in the plot shown by check\_alignment\_consistency() uses the same labelling.

If you choose to add, remove or change data or changing the q-range after doing an alignment, redo the alignment with the
setting redo\_import=True.
Else it will not respect the changes you made. If you just want to change the number of interations or the regrouping,
this is not necessary.

\subsection{Model}
Next you have to calculate the model, which consists of 2 parts: Diffractlets and Projectors.

Diffractlets are calculated from the crystal structure given by a .cif file, you have to provide.
When you start the model calculation using make\_model(), you will receive another file to edit (crystal.py), containing
information about the location of your .cif file, X-ray energy, q-range and desired angular resolution.
Save the file and it will be copied to sampel\_dir/analysis/crystal.py.
The function will create the file crystal.h5, containing the diffractlets. As this calculation can be lengthy,
it is advised to perform it in advance and reuse crystal.h5 for other samples. If sample\_dir/analysis/ contains
already a crystal.h5 file, it will use this without asking.

The projectors contain information on which voxels contribute to which pixel in the data and
depend on a finished alignment. Once you finished the alignment you can start calculating the projectors,
which requires some more user input for masking the sample. The program will open a histogram of voxels based on the
tomogram resulting from alignment. Choose the lower cutoff to mask out voxels with low or zero density of crystallites,
upon which you will be shown a 3D outline of the sample.
You can remove other parts of the sample using the input in the figure.
After processing, this will create a file analysis/tomo.h5, which is used in further processing of this specific sample.

\subsection{Data Pre-processing}

When the model is ready, the data has to pass through a pre-processing step, where it is filtered according to
which data is masked, then renormalized and outliers are removed. You will be also asked to choose the q-ranges
around the peaks you would like to use for optimization, and to define the detector mask.
Text files will be created, these can be re-used for other samples and will be automatically chosen if present
in the analysis/ directory.
There is also a simple background subtraction pipeline, which can be turned on using the argument draw\_baselines=True. 
Note that this feature is still experimental and might not work with every sample.



\subsection{Optimization}
If all previous steps have been performed, you can start an optimization.


\subsection{Visualisation}


\section{Functions}
\section*{Module: textom.textom}

\subsection*{\texttt{set\_path(path)}}
\addcontentsline{toc}{subsection}{set\_path}

\begin{lstlisting}[language=docstring]
Set the path where integrated data and analysis is stored

Parameters
----------
path : str
    full path to the directory, must contain a folder '/data_integrated'
\end{lstlisting}

\begin{flushright}

\hyperref[toc]{ToC}

\end{flushright}

\input{functions/set_path}

\vspace{5mm}

\hrule

\subsection*{\texttt{check\_state()}}
\addcontentsline{toc}{subsection}{check\_state}

\begin{lstlisting}[language=docstring]
Prints in terminal which parts of the reconstruction are ready
    
\end{lstlisting}

\begin{flushright}

\hyperref[toc]{ToC}

\end{flushright}

\input{functions/check_state}

\vspace{5mm}

\hrule

\subsection*{\texttt{integrate()}}
\addcontentsline{toc}{subsection}{integrate}

\begin{lstlisting}[language=docstring]
Starts integrating raw data via pyFAI
    
\end{lstlisting}

\begin{flushright}

\hyperref[toc]{ToC}

\end{flushright}

\input{functions/integrate}

\vspace{5mm}

\hrule

\subsection*{\texttt{align\_data(pattern='.h5', sub\_data='data\_integrated', q\_index\_range=(0, 5), q\_range=False, crop\_image=False, mode='optical\_flow', redo\_import=False, flip\_fov=False, regroup\_max=16, align\_horizontal=True, align\_vertical=True, pre\_rec\_it=5, pre\_max\_it=5, last\_rec\_it=40, last\_max\_it=5)}}
\addcontentsline{toc}{subsection}{align\_data}

\begin{lstlisting}[language=docstring]
Align data using the Mumott optical flow alignment

Requires that data has been integrated and that sample_dir contains
a subfolder with data

Parameters
----------
pattern : str, optional
    substring contained in all files you want to use, by default '.h5'
sub_data : str, optional
    subfolder containing the data, by default 'data_integrated'
q_index_range : tuple, optional
    determines which q-values are used for alignment (sums over them), by default (0,5)
q_range : tuple, optional
    give the q-range in nm instead of indices e.g. (15.8,18.1), by default False
crop_image : bool or tuple of int, optional
    give the range you want to use in x and y, e.g. ((0,-1),(10,-10)), by default False
mode : str, optional
    choose alignment mode, 'optical_flow' or 'phase_matching', by default 'optical_flow'
redo_import : bool, optional
    set True if you want to recalculate data_mumott.h5, by default False
flip_fov : bool, optional
    only to be used if the fov is in the wrong order in the integrated
    data files, by default False
regroup_max : int, optional
    maximum size of groups when downsampling for faster processing, by default 16
align_horizontal : bool, optional
    align your data horizontally, by default True
align_vertical : bool, optional
    align your data vertically, by default True
pre_rec_it : int, optional
    reconstruciton iterations for downsampled data, by default 5
pre_max_it : int, optional
    alignment iterations for downsampled data, by default 5
last_rec_it : int, optional
    reconstruciton iterations for full data, by default 40
last_max_it : int, optional
    alignment iterations for full data, by default 5
\end{lstlisting}

\begin{flushright}

\hyperref[toc]{ToC}

\end{flushright}

\input{functions/align_data}

\vspace{5mm}

\hrule

\subsection*{\texttt{check\_alignment\_consistency()}}
\addcontentsline{toc}{subsection}{check\_alignment\_consistency}

\begin{lstlisting}[language=docstring]
Plots the squared residuals between data and the projected tomograms

    
\end{lstlisting}

\begin{flushright}

\hyperref[toc]{ToC}

\end{flushright}

\input{functions/check_alignment_consistency}

\vspace{5mm}

\hrule

\subsection*{\texttt{check\_alignment\_projection(g=0)}}
\addcontentsline{toc}{subsection}{check\_alignment\_projection}

\begin{lstlisting}[language=docstring]
Plots the data and the projected tomogram of projection g

Parameters
----------
g : int, optional
    projection running index, by default 0
\end{lstlisting}

\begin{flushright}

\hyperref[toc]{ToC}

\end{flushright}

\input{functions/check_alignment_projection}

\vspace{5mm}

\hrule

\subsection*{\texttt{make\_model()}}
\addcontentsline{toc}{subsection}{make\_model}

\begin{lstlisting}[language=docstring]
Calculates the TexTOM model for reconstructions

Is automatically performed by the functions that require it
\end{lstlisting}

\begin{flushright}

\hyperref[toc]{ToC}

\end{flushright}

\input{functions/make_model}

\vspace{5mm}

\hrule

\subsection*{\texttt{preprocess\_data(pattern='.h5', flip\_fov=False, baselines=True, use\_ion=True)}}
\addcontentsline{toc}{subsection}{preprocess\_data}

\begin{lstlisting}[language=docstring]
Loads integrated data and pre-processes them for TexTOM

Parameters
----------
pattern : str, optional
    substring contained in all files you want to use, by default '.h5'
flip_fov : bool, optional
    only to be used if the fov is in the wrong order in the integrated
    data files, by default False
baselines : bool, optional
    choose if you want to draw polynomial baselines, by default True
use_ion : bool, optional
    choose if you want to normalize data by the field 'ion' in the 
    data files, by default True
\end{lstlisting}

\begin{flushright}

\hyperref[toc]{ToC}

\end{flushright}

\input{functions/preprocess_data}

\vspace{5mm}

\hrule

\subsection*{\texttt{make\_fit(redo=True)}}
\addcontentsline{toc}{subsection}{make\_fit}

\begin{lstlisting}[language=docstring]
Initializes a TexTOM fit object for reconstructions

Is automatically performed by the functions that require it

Parameters
----------
redo : bool, optional
    set True for recalculating, by default True
\end{lstlisting}

\begin{flushright}

\hyperref[toc]{ToC}

\end{flushright}

\input{functions/make_fit}

\vspace{5mm}

\hrule

\subsection*{\texttt{optimize(order=0, mode=0, proj='full', redo\_fit=False, tol=0.001, minstep=1e-09, itermax=3000, alg='quadratic', save\_h5=True)}}
\addcontentsline{toc}{subsection}{optimize}

\begin{lstlisting}[language=docstring]
Performs a single TexTOM parameter optimization

Parameters
----------
order : int, optional
    maximum sHSH order to be used, by default 0
mode : int, optional
    set 0 for only optimizing order 0, 1 for highest order, 2 for all,
    by default 0
proj : str, optional
    choose projections to be optimized: 'full', 'half', 'third', 'notilt', 
    by default 'full'
redo_fit : bool, optional
    recalculate the fit object, by default False
tol : float, optional
    tolerance for precision break criterion, by default 1e-3
minstep : float, optional
    minimum stepsize in line search, by default 1e-9
itermax : int, optional
    maximum number of iterations, by default 3000
alg : str, optional
    choose algorithm between 'backtracking', 'simple', 'quadratic', 
    by default 'quadratic'
save_h5 : bool, optional
    choose if you want to save the result to the directory analysis/fits, 
    by default True    
\end{lstlisting}

\begin{flushright}

\hyperref[toc]{ToC}

\end{flushright}

\input{functions/optimize}

\vspace{5mm}

\hrule

\subsection*{\texttt{optimize\_auto(max\_order=8, start\_order=None, tol\_0=1e-07, tol\_1=0.001, tol\_2=0.0001, minstep\_0=1e-09, minstep\_1=1e-09, minstep\_2=1e-09, projections='full', alg='quadratic', adj\_scal=False, redo\_fit=False)}}
\addcontentsline{toc}{subsection}{optimize\_auto}

\begin{lstlisting}[language=docstring]
Automated TexTOM reconstruction workflow

Parameters
----------
max_order : int, optional
    maximum HSH order to be used, by default 8    
start_order : int or None, optional
    lowest order to be fitted, if None continues where you are standing, 
    by default None
redo_fit : bool, optional
    recalculate the fit object, by default False
proj : str, optional
    choose projections to be optimized: 'full', 'half', 'third', 'notilt', by default 'full'
alg : str, optional
    choose algorithm between 'backtracking', 'simple', 'quadratic', 
    by default 'quadratic'
\end{lstlisting}

\begin{flushright}

\hyperref[toc]{ToC}

\end{flushright}

\input{functions/optimize_auto}

\vspace{5mm}

\hrule

\subsection*{\texttt{list\_opt()}}
\addcontentsline{toc}{subsection}{list\_opt}

\begin{lstlisting}[language=docstring]
Shows all stored optimizations
    
\end{lstlisting}

\begin{flushright}

\hyperref[toc]{ToC}

\end{flushright}

\input{functions/list_opt}

\vspace{5mm}

\hrule

\subsection*{\texttt{load\_opt(h5path='last')}}
\addcontentsline{toc}{subsection}{load\_opt}

\begin{lstlisting}[language=docstring]
Loads a previous Textom optimization into memory
seful: load_opt(results['optimization'])

Parameters
----------
h5path : str, optional
    filepath, just filename or full path
    if 'last', uses the youngest file is used in analysis/fits/, 
    by default 'last'
\end{lstlisting}

\begin{flushright}

\hyperref[toc]{ToC}

\end{flushright}

\input{functions/load_opt}

\vspace{5mm}

\hrule

\subsection*{\texttt{check\_lossfunction()}}
\addcontentsline{toc}{subsection}{check\_lossfunction}

\begin{lstlisting}[language=docstring]
No docstring available.
\end{lstlisting}

\begin{flushright}

\hyperref[toc]{ToC}

\end{flushright}

\input{functions/check_lossfunction}

\vspace{5mm}

\hrule

\subsection*{\texttt{check\_fit\_average()}}
\addcontentsline{toc}{subsection}{check\_fit\_average}

\begin{lstlisting}[language=docstring]
Plots the reconstructed average intensity for each projection with data

Parameters
----------
\end{lstlisting}

\begin{flushright}

\hyperref[toc]{ToC}

\end{flushright}

\input{functions/check_fit_average}

\vspace{5mm}

\hrule

\subsection*{\texttt{check\_fit\_random(N=10, mode='line')}}
\addcontentsline{toc}{subsection}{check\_fit\_random}

\begin{lstlisting}[language=docstring]
Generates TexTOM reconstructions and plots them with data for random points

Parameters
----------
N : int, optional
    Number of images created, by default 10    
mode : str, optional
    plotting mode, 'line' or 'color', by default line
\end{lstlisting}

\begin{flushright}

\hyperref[toc]{ToC}

\end{flushright}

\input{functions/check_fit_random}

\vspace{5mm}

\hrule

\subsection*{\texttt{check\_residuals()}}
\addcontentsline{toc}{subsection}{check\_residuals}

\begin{lstlisting}[language=docstring]
Plots the squared residuals summed over each projection
    
\end{lstlisting}

\begin{flushright}

\hyperref[toc]{ToC}

\end{flushright}

\input{functions/check_residuals}

\vspace{5mm}

\hrule

\subsection*{\texttt{check\_projections\_average(G=None)}}
\addcontentsline{toc}{subsection}{check\_projections\_average}

\begin{lstlisting}[language=docstring]
Plots the reconstructed average intensity for chosen projections with data

Parameters
----------
G : int or ndarray or None, optional
    projection indices, if None takes 10 equidistant ones, by default None
\end{lstlisting}

\begin{flushright}

\hyperref[toc]{ToC}

\end{flushright}

\input{functions/check_projections_average}

\vspace{5mm}

\hrule

\subsection*{\texttt{check\_projections\_residuals(G=None)}}
\addcontentsline{toc}{subsection}{check\_projections\_residuals}

\begin{lstlisting}[language=docstring]
Plots the residuals per pix3l for chosen projections with data

Parameters
----------
G : int or ndarray or None, optional
    projection indices, if None takes 10 equidistant ones, by default None
\end{lstlisting}

\begin{flushright}

\hyperref[toc]{ToC}

\end{flushright}

\input{functions/check_projections_residuals}

\vspace{5mm}

\hrule

\subsection*{\texttt{check\_projections\_orientations(G=None)}}
\addcontentsline{toc}{subsection}{check\_projections\_orientations}

\begin{lstlisting}[language=docstring]
Plots the reconstructed average orientations for chosen projections with data

Parameters
----------
G : int or ndarray or None, optional
    projection indices, if None takes 10 equidistant ones, by default None
\end{lstlisting}

\begin{flushright}

\hyperref[toc]{ToC}

\end{flushright}

\input{functions/check_projections_orientations}

\vspace{5mm}

\hrule

\subsection*{\texttt{calculate\_orientation\_statistics()}}
\addcontentsline{toc}{subsection}{calculate\_orientation\_statistics}

\begin{lstlisting}[language=docstring]
Calculates prefered orientations and stds and saves them to results dict

    
\end{lstlisting}

\begin{flushright}

\hyperref[toc]{ToC}

\end{flushright}

\input{functions/calculate_orientation_statistics}

\vspace{5mm}

\hrule

\subsection*{\texttt{calculate\_segments(thresh=10, min\_segment\_size=30, max\_segments\_number=31)}}
\addcontentsline{toc}{subsection}{calculate\_segments}

\begin{lstlisting}[language=docstring]
Segments the sample based on misorientation borders

Parameters
----------
thresh : float, optional
    misorientation angle threshold inside segment in degree, by default 10
min_segment_size : int, optional
    minimum number of voxels in segment, by default 30
max_segments_number : int, optional
    maximum number of segments (ordered by size), by default 32
\end{lstlisting}

\begin{flushright}

\hyperref[toc]{ToC}

\end{flushright}

\input{functions/calculate_segments}

\vspace{5mm}

\hrule

\subsection*{\texttt{show\_volume(data='scaling', plane='z', colormap='inferno', cut=1, save=False, show=True)}}
\addcontentsline{toc}{subsection}{show\_volume}

\begin{lstlisting}[language=docstring]
Visualizes the whole sample by slices, colored by a value of your choice

Parameters
----------
data : str or list, optional
    name of one entry in the results dict or list of entries, 
    by default 'scaling'
plane : str, optional
    sliceplane 'x'/'y'/'z', by default 'z'
colormap : str, optional
    identifier of matplotlib colormap, default 'inferno'
    https://matplotlib.org/stable/users/explain/colors/colormaps.html
cut : int, optional
    cut colorscale at upper and lower percentile, by default 0.1
\end{lstlisting}

\begin{flushright}

\hyperref[toc]{ToC}

\end{flushright}

\input{functions/show_volume}

\vspace{5mm}

\hrule

\subsection*{\texttt{show\_slice\_ipf(h, plane='z')}}
\addcontentsline{toc}{subsection}{show\_slice\_ipf}

\begin{lstlisting}[language=docstring]
Plots an inverse pole figure of a sample slice

Parameters
----------
h : int
    height of the slice
plane : str, optional
    slice direction: x/y/z, by default 'z'
\end{lstlisting}

\begin{flushright}

\hyperref[toc]{ToC}

\end{flushright}

\input{functions/show_slice_ipf}

\vspace{5mm}

\hrule

\subsection*{\texttt{show\_volume\_ipf(plane='z', save=False, show=True)}}
\addcontentsline{toc}{subsection}{show\_volume\_ipf}

\begin{lstlisting}[language=docstring]
Plots inverse pole figures as a tomogram with a slider to scroll through the sample

Parameters
----------
plane : str, optional
    slice direction: x/y/z, by default 'z'
save : bool, optional
    if True, saves movie as .gif to results/images, by default False
show : bool, optional
    if True, opens matplotlib window, by default True
\end{lstlisting}

\begin{flushright}

\hyperref[toc]{ToC}

\end{flushright}

\input{functions/show_volume_ipf}

\vspace{5mm}

\hrule

\subsection*{\texttt{show\_histogram(x, nbins=50, cut=0.1, segments=None, save=False)}}
\addcontentsline{toc}{subsection}{show\_histogram}

\begin{lstlisting}[language=docstring]
plots a histogram of a result parameter

Parameters
----------
x : str,
    name of a scalar from results
bins : int, optional
    number of bins, by default 50
cut : int, optional
    cut upper and lower percentile, by default 0.1
segments : list of int, optional
    list of segments or None for all data, by default None
\end{lstlisting}

\begin{flushright}

\hyperref[toc]{ToC}

\end{flushright}

\input{functions/show_histogram}

\vspace{5mm}

\hrule

\subsection*{\texttt{show\_correlations(x, y, nbins=50, cut=(0.1, 0.1), segments=None, save=False)}}
\addcontentsline{toc}{subsection}{show\_correlations}

\begin{lstlisting}[language=docstring]
plots a 2D histogram between 2 result parameters

Parameters
----------
x : str,
    name of a scalar from results
y : str,
    name of a scalar from results
bins : int, optional
    number of bins, by default 50
cut : tuple, optional
    cut upper and lower percentile of both parameters, by default (0.1,0.1)
segments : list, optional
    list of segments or None for all data, by default None
\end{lstlisting}

\begin{flushright}

\hyperref[toc]{ToC}

\end{flushright}

\input{functions/show_correlations}

\vspace{5mm}

\hrule

\subsection*{\texttt{show\_voxel\_odf(x, y, z, num\_samples=1000)}}
\addcontentsline{toc}{subsection}{show\_voxel\_odf}

\begin{lstlisting}[language=docstring]
Show a 3D plot of the ODF in the chosen voxel

Parameters
----------
x : int
    voxel x-coordinate
y : int
    voxel y-coordinate
z : int
    voxel z-coordinate
\end{lstlisting}

\begin{flushright}

\hyperref[toc]{ToC}

\end{flushright}

\input{functions/show_voxel_odf}

\vspace{5mm}

\hrule

\subsection*{\texttt{show\_voxel\_polefigure(x, y, z, hkl=(1, 0, 0), mode='density', alpha=0.1, num\_samples=10000.0)}}
\addcontentsline{toc}{subsection}{show\_voxel\_polefigure}

\begin{lstlisting}[language=docstring]
Show a polefigure plot for the chosen voxel and hkl

Parameters
----------
x : int
    voxel x-coordinate
y : int
    voxel y-coordinate
z : int
    voxel z-coordinate
hkl : tuple, optional
    Miller indices, by default (1,0,0)
mode : str, optional
    plotting style 'scatter' or 'density', by default 'density'
alpha : float, optional
    opacity of points, only for scatter, by default 0.1
num_samples : int/float, optional
    number of samples for plot generation, by default 1e4
\end{lstlisting}

\begin{flushright}

\hyperref[toc]{ToC}

\end{flushright}

\input{functions/show_voxel_polefigure}

\vspace{5mm}

\hrule

\subsection*{\texttt{reconstruct\_1d\_full(redo\_import=False, only\_mumottize=False, batch\_size=10)}}
\addcontentsline{toc}{subsection}{reconstruct\_1d\_full}

\begin{lstlisting}[language=docstring]
Reconstructs standard tomographic data such as azimutally averaged
diffraction data. Uses the same alignment as textom

Parameters
----------
redo_import : bool, optional
    _description_, by default False
only_mumottize : bool, optional
    only preprocesses a file analysis/rec1d/data_rec1d.h5, by default False
batch_size : int, optional
    number of q-values to load at the same time. Needs to be an integer fraction
    of the total number of q-values, else it will crash at the last batch. Higher
    numbers will decrease i/o time, but require more memory, by default 10
\end{lstlisting}

\begin{flushright}

\hyperref[toc]{ToC}

\end{flushright}

\input{functions/reconstruct_1d_full}

\vspace{5mm}

\hrule

\subsection*{\texttt{save\_results()}}
\addcontentsline{toc}{subsection}{save\_results}

\begin{lstlisting}[language=docstring]
Saves the results dictionary to a h5 file

    
\end{lstlisting}

\begin{flushright}

\hyperref[toc]{ToC}

\end{flushright}

\input{functions/save_results}

\vspace{5mm}

\hrule

\subsection*{\texttt{link\_xdmf(paths)}}
\addcontentsline{toc}{subsection}{link\_xdmf}

\begin{lstlisting}[language=docstring]
No docstring available.
\end{lstlisting}

\begin{flushright}

\hyperref[toc]{ToC}

\end{flushright}

\input{functions/link_xdmf}

\vspace{5mm}

\hrule

\subsection*{\texttt{list\_results()}}
\addcontentsline{toc}{subsection}{list\_results}

\begin{lstlisting}[language=docstring]
Shows all results .h5 files in results directory
    
\end{lstlisting}

\begin{flushright}

\hyperref[toc]{ToC}

\end{flushright}

\input{functions/list_results}

\vspace{5mm}

\hrule

\subsection*{\texttt{load\_results(h5path='last', make\_bg\_nan=False)}}
\addcontentsline{toc}{subsection}{load\_results}

\begin{lstlisting}[language=docstring]
Loads the results from a h5 file do the results dictionary

    
\end{lstlisting}

\begin{flushright}

\hyperref[toc]{ToC}

\end{flushright}

\input{functions/load_results}

\vspace{5mm}

\hrule

\subsection*{\texttt{list\_results\_loaded()}}
\addcontentsline{toc}{subsection}{list\_results\_loaded}

\begin{lstlisting}[language=docstring]
Shows all results currently in memory
    
\end{lstlisting}

\begin{flushright}

\hyperref[toc]{ToC}

\end{flushright}

\input{functions/list_results_loaded}

\vspace{5mm}

\hrule

\subsection*{\texttt{save\_images(x, ext='raw')}}
\addcontentsline{toc}{subsection}{save\_images}

\begin{lstlisting}[language=docstring]
Export results as .raw or .tiff files for dragonfly

Parameters
----------
x : str,
    name of a scalar from results, e.g. 'scaling'
\end{lstlisting}

\begin{flushright}

\hyperref[toc]{ToC}

\end{flushright}

\input{functions/save_images}

\vspace{5mm}

\hrule

\subsection*{\texttt{sort\_integrated\_files(source='data\_integrated', target='data\_integrated\_1d', check\_end=-6, pattern='.h5')}}
\addcontentsline{toc}{subsection}{sort\_integrated\_files}

\begin{lstlisting}[language=docstring]
Reads all filenames in source folder and checks if the same exist in target
Moves the non-overlapping to an 'excluded'-subfolder 

Parameters
----------
source : str, optional
    directory with sorted files, by default 'data_integrated'
target : str, optional
    directory with files to sort, by default 'data_integrated_1d'
check_end : int, optional
    index of the last character of the filename to check, by default -6
pattern : str, optional
    only filenames that contain this are searched, by default '.h5'
\end{lstlisting}

\begin{flushright}

\hyperref[toc]{ToC}

\end{flushright}

\input{functions/sort_integrated_files}

\vspace{5mm}

\hrule

\subsection*{\texttt{help(method=None, module=None, filter='')}}
\addcontentsline{toc}{subsection}{help}

\begin{lstlisting}[language=docstring]
Prints information about functions in this library

Parameters
----------
method : str or None, optional
    get more information about a function or None for overview over all functions, by default None
module : str or None, optional
    choose python module or None for the base TexTOM library, by default None
filter : str, optional
    filter the displayed functions by a substring, by default ''
\end{lstlisting}

\begin{flushright}

\hyperref[toc]{ToC}

\end{flushright}

\input{functions/help}

\vspace{5mm}

\hrule

\end{document}\end{document}