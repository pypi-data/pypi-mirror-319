\chapter{Overview}

\tao is an open source general purpose program for charged particle and X-ray simulations in
accelerators and storage rings. It is built on top of the \bmad toolkit (software library) which
provides the needed computational routines needed to do simulations. Essentially you can think of
\tao as a car and \bmad as the engine that powers the car. In fact \bmad powers a number of other
simulation programs but that is getting outside of the scope of this manual. 

Documentation for \bmad and \tao, as well as information for downloading the code if needed is given
on the \bmad web site
\hfill\break
\hspace*{0.3in} \url{https://www.classe.cornell.edu/bmad}

\tao by itself is a command line program. To make it more readily accessible, a graphic user
interface (GUI) has been developed and this is the subject of this manual. The GUI is written in
Python. The Python \vn{tkinter} package is used for windowing. \vn{Tkinter} being an interface layer
for the \vn{Tk} widget toolset. The Python based plotting package MatPlotLib can be used for graphics.
Alternatively, the PGPLOT/PLPLOT plotting that comes standard with Tao can be used.

It is assumed in this manual that the reader already has some familiarity with \bmad and \tao. If not,
there are manuals for \bmad and \tao posted on the \bmad web site along with a beginner's tutorial.

The coding the of GUI was a joint development project with John Mastroberti, Kevin Kowalski, and
David Sagan. Thanks must go to Thomas Gl{\"a}{\ss}e for helping with the inital development.


