%%%%%%%%%%%%%%%%%%%%%%%%%%%%%%%%%%%%%%%%%%%%%%%%%%%%%%%%%%%%%%%%%%%%%%%%%
%%
%A  underl.tex         AutPGrp documentation                 Bettina Eick
%A                                                         Eamonn O'Brien
%%

%%%%%%%%%%%%%%%%%%%%%%%%%%%%%%%%%%%%%%%%%%%%%%%%%%%%%%%%%%%%%%%%%%%%%%%%%
\Chapter{The underlying function}

Underlying the method installation for `AutomorphismGroup'
is the function `AutomorphismGroupPGroup'. This function is
intended for expert users who wish to influence the steps of 
the algorithm.  Note also that `AutomorphismGroup' will always
choose default values.

\> AutomorphismGroupPGroup( <G> [,<flag>] )  F

The input is a finite $p$-group as above and an optional <flag> 
which can be true or false. Here the filters for <G> need not be 
set, but they should be true for <G>. The possible values for <flag>
are considered later in Chapter "Influencing the algorithm". If 
<flag> is not supplied, the algorithm proceeds similarly to the 
method installed for `AutomorphismGroup', but it produces slightly 
more detailed output.  The output of the function is a record 
which contains the following fields:

\beginitems
`glAutos' & a set of automorphisms which together with `agAutos'
            generate the automorphism group;

`glOrder' & an integer whose product with the `agOrders' gives
            the size of the automorphism group;

`agAutos' & a polycyclic generating sequence for a soluble normal
            subgroup of the automorphism group;

`agOrder' & the relative orders corresponding to `agAutos';

`one'     & the identity element of the automorphism group;

`group'   & the underlying group <G>;

`size'    & the size of the automorphism group.
\enditems

We do not return an automorphism group in the standard form 
because we wish to distinguish between `agAutos' and `glAutos'; 
the latter act non-trivially on the Frattini quotient of <G>. This
hybrid-group description of the automorphism group permits more 
efficient computations with it. The following function converts
the output of `AutomorphismGroupPGroup' to the output of 
`AutomorphismGroup'.

\> ConvertHybridAutGroup( <A> ) F

\beginexample 
gap> LoadPackage("autpgrp", false);
true

gap> H := SmallGroup (729, 34);
<pc group of size 729 with 6 generators>

gap> A := AutomorphismGroupPGroup(H);
rec( glAutos := [  ], 
     glOrder := 1, 
     agAutos := [ Pcgs([ f1, f2, f3, f4, f5, f6 ]) 
                    -> [ f1^2, f2, f3^2*f4, f4, f5^2*f6, f6 ], 
                  Pcgs([ f1, f2, f3, f4, f5, f6 ]) 
                    -> [ f2^2, f1, f3*f5^2, f5^2, f4*f6^2, f6^2 ], 
                  Pcgs([ f1, f2, f3, f4, f5, f6 ]) 
                    -> [ f1^2, f2^2, f3*f4^2*f5^2*f6, f4^2*f6, f5^2*f6, f6 ], 
                  Pcgs([ f1, f2, f3, f4, f5, f6 ]) 
                    -> [ f1*f3, f2, f3*f5^2, f4*f6^2, f5, f6 ], 
                  Pcgs([ f1, f2, f3, f4, f5, f6 ])
                    -> [ f1, f2*f3, f3*f4, f4, f5*f6, f6 ], 
                  Pcgs([ f1, f2, f3, f4, f5, f6 ]) 
                    -> [ f1*f4, f2, f3*f6^2, f4, f5, f6 ], 
                  Pcgs([ f1, f2, f3, f4, f5, f6 ]) 
                    -> [ f1, f2*f4, f3, f4, f5, f6 ], 
                  Pcgs([ f1, f2, f3, f4, f5, f6 ]) 
                    -> [ f1*f5, f2, f3, f4, f5, f6 ], 
                  Pcgs([ f1, f2, f3, f4, f5, f6 ]) 
                    -> [ f1, f2*f5, f3*f6, f4, f5, f6 ], 
                  Pcgs([ f1, f2, f3, f4, f5, f6 ]) 
                    -> [ f1*f6, f2, f3, f4, f5, f6 ], 
                  Pcgs([ f1, f2, f3, f4, f5, f6 ]) 
                    -> [ f1, f2*f6, f3, f4, f5, f6 ] ], 
     agOrder := [ 2, 2, 2, 3, 3, 3, 3, 3, 3, 3, 3 ], 
     one     := IdentityMapping( <pc group of size 729 with 6 generators> ), 
     group   := <pc group of size 729 with 6 generators>, 
     size    := 52488 )

gap> ConvertHybridAutGroup( A );
<group of size 52488 with 11 generators>
\endexample 

Let <A> be the automorphism group of a $p$-group $G$ as computed by 
`AutomorphismGroupPGroup'. Then the following function can compute 
a pc group isomorphic to the solvable part of <A> stored in the record 
component <A>.agGroup. This solvable part forms a subgroup of the
automorphism group which  contains at least the automorphisms centralizing
the Frattini factor of $G$. The pc group facilitates various further
computations with <A>.

\> PcGroupAutPGroup( <A> ) F

computes a pc presentation for the solvable part of the automorphism
group <A> defined by <A>.agGroup. <A> is the output of the function
`AutomorphismGroupPGroup'.

\beginexample
gap> H := SmallGroup (729, 34);;
gap> A := AutomorphismGroupPGroup(H);;
gap> B := PcGroupAutPGroup( A );
<pc group of size 52488 with 11 generators>
gap> I := InnerAutGroupPGroup( B );
Group([ f5, f4^2*f8, f6^2*f9^2, f11^2, f10^2, <identity> of ... ])
\endexample

%%%%%%%%%%%%%%%%%%%%%%%%%%%%%%%%%%%%%%%%%%%%%%%%%%%%%%%%%%%%%%%%%%%%%%%%%%
